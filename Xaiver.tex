%==============================常用宏包、环境==============================%
\documentclass[landscape,twocolumn,a4paper]{article}
\usepackage{xeCJK} % For Chinese characters
\usepackage{amsmath, amsthm}
\usepackage{listings,xcolor}
\usepackage{geometry} % 设置页边距
\usepackage{fontspec}
\usepackage{graphicx}
\usepackage{fancyhdr} % 自定义页眉页脚
\usepackage[breaklinks,colorlinks,linkcolor=black,citecolor=black,urlcolor=black]{hyperref}
\setsansfont{Consolas} % 设置英文字体
\setmonofont[Mapping={}]{Consolas} % 英文引号之类的正常显示,相当于设置英文字体
\geometry{left=1cm,right=1cm,top=2cm,bottom=0.5cm} % 页边距
\setlength{\columnsep}{30pt}
% \setlength\columnseprule{0.4pt} % 分割线

\usepackage{xunicode, xltxtra} 
\setmainfont{Microsoft YaHei} 
\usepackage{setspace}
\usepackage{ctex} 
\usepackage[Glenn]{fncychap}
\usepackage{color}
\usepackage{verbatim}



%==============================常用宏包、环境==============================%
\newfontfamily\monaco{Monaco}
\definecolor{dkgreen}{rgb}{0,0.6,0}
\definecolor{gray}{rgb}{0.5,0.5,0.5}
\definecolor{mauve}{rgb}{0.58,0,0.82}

%==============================页眉、页脚、代码格式设置==============================%
% 页眉、页脚设置
\pagestyle{fancy}
% \lhead{CUMTB}
\lhead{\CJKfamily{hei} Standard Code Library}
\chead{}
% \rhead{Page \thepage}
\rhead{\CJKfamily{hei} 第 \thepage 页}
\lfoot{} 
\cfoot{}
\rfoot{}
\renewcommand{\headrulewidth}{0.4pt} 
\renewcommand{\footrulewidth}{0.4pt}

% 代码格式设置
% \lstset{
%     frame=simple,
%     language=c++,
%     aboveskip=3mm,
%     belowskip=3mm,
%     showstringspaces=false,
%     basicstyle=\monaco,
%     numbers=none,
%     numberstyle=\tiny\color{gray},
%     keywordstyle=\bfseries\monaco,%\fontspec{monaco Bold}\bfseries,
%     commentstyle=\color{dkgreen},
%     stringstyle=\color{mauve},
%     breaklines=true,
%     breakatwhitespace=true,
%     tabsize=4,
%     numbers = left,
% }

\lstset{
    language    = c++,
    numbers     = left,
    numberstyle = \tiny,
    breaklines  = true,
    captionpos  = b,
    tabsize     = 4,
    frame       = simple,
    columns     = fullflexible,
    commentstyle = \color[RGB]{0,128,0},
    keywordstyle = \color[RGB]{0,0,255},
    basicstyle   = \monaco,
    stringstyle  = \color[RGB]{148,0,209}\ttfamily,
    rulesepcolor = \color{red!20!green!20!blue!20},
    showstringspaces = false,
}
%==============================页眉、页脚、代码格式设置==============================%

%==============================标题和目录==============================%
\title{\CJKfamily{hei} \bfseries Standard Code Library}
\author{Xavier\_Cai}
\renewcommand{\today}{\number\year 年 \number\month 月 \number\day 日}

\begin{document}\small
\begin{titlepage}
\maketitle
\end{titlepage}

\newpage
\pagestyle{empty}
\renewcommand{\contentsname}{目录}
\tableofcontents
\newpage\clearpage
\newpage
\pagestyle{fancy}
\setcounter{page}{1}   %new page
%==============================标题和目录==============================%
%==============================正文部分==============================%
%\begin{lstlisting}
%\end{lstlisting}
%\lstinputlisting{code/sieve.txt}
%==============================数据结构==============================%
\section{数据结构}

\subsection{LCT}
\subsubsection{lct}
\lstinputlisting{数据结构/LCT/lct.cpp}
\subsubsection{树上路径染色}
\lstinputlisting{数据结构/LCT/树上路径染色([SDOI2011]染色).cpp}
\subsubsection{离线统计区间本质不同字串个数}
时间复杂度:$O(n \log^{2}n + m \log n)$.
\lstinputlisting{数据结构/LCT/SAM+线段树+LCT离线统计区间本质不同字串个数.cpp}
\subsubsection{在线查询边的区间内连通块个数}
时间复杂度:$O(m \log n + q \log m)$.
\lstinputlisting{数据结构/LCT/主席树+LCT在线查询区间连通块个数.cpp}

\subsection{有旋Treap \&无旋Treap}
\subsubsection{带跳fa的Treap([ZJOI2006]书架)}
\lstinputlisting{数据结构/无旋Treap/带跳fa的Treap([ZJOI2006]书架).cpp}
\subsubsection{并查集+启发式合并(HDU3726)}
\lstinputlisting{数据结构/有旋Treap/并查集+启发式合并(HDU3726).cpp}

\subsection{李超线段树}
\subsubsection{多条线段定点最值([HEOI2013]Segment)}
要求在平面直角坐标系下维护两个操作(强制在线):\par
\textbf{1 x0 y0 x1 y1} 在平面上加入一条线段。\par
\textbf{0 x} 给定一个数,询问与直线$y=x$相交的线段中,交点纵坐标最大的线段的编号(若有多条线段与查询直线的交点纵坐标都是最大的,则输出编号最小的线段)。特别地,若不存在线段与给定直线相交,输出0。
\lstinputlisting{数据结构/李超线段树/函数定点最值([HEOI2013]Segment).cpp}

\subsection{YNOI系列}
\subsubsection{带修查询能否连续重排为值域连续的序列(线段树+散列异或)(洛谷P3792)}
\textbf{1 x y} 修改$x$位置的值为$y$\par
\textbf{2 l r} 查询区间$[l,r]$是否可以重排为值域上连续的一段\par
\lstinputlisting {数据结构/YNOI系列/带修查询能否连续重排为值域连续的序列(线段树+散列异或)(洛谷P3792).cpp}

%==============================字符串==============================%
% \newpage\clearpage
% \newpage

\section{字符串}

\subsection{后缀连接字典序最小}
\lstinputlisting{字符串/后缀数组/习题整理/后缀连接字典序最小(arc050_d).cpp}

%==============================动态规划==============================%

\section{动态规划}

\subsection{\#2 字符串T在字符串S子序列出现的次数}
\lstinputlisting{动态规划/2_字符串T在字符串S子序列出现的次数.cpp}
\subsection{\#3 N种长度为1元素填充L}
\lstinputlisting{动态规划/3_N种长度为1元素填充L.cpp}
\subsection{\#4 分割数组}
\lstinputlisting{动态规划/4_分割数组.cpp}
\subsection{\#5 划分为K个相等的子集}
\lstinputlisting{动态规划/5_划分为K个相等的子集.cpp}

%==============================杂项==============================%

\section{杂项}
\subsection{散列处理异或碰撞}
常用于处理判断出现偶数次(HDU6291),重排后为值域上的连续一段(洛谷P3792、YNOI)。
\lstinputlisting{其它/散列处理异或碰撞.cpp}

\subsection{fread}
\lstinputlisting{杂项/fread.cpp}

%==============================习题整理==============================%

\section{习题整理}
\subsection{图上加边最多最少连通块(线段树二分贪心)(ZOJ4100)}
时间复杂度:$O(q \log ^ {2} n)$.
\lstinputlisting{习题整理/图上加边最多最少连通块(线段树二分贪心).cpp}
\subsection{错排后字典序最小(ZOJ4102)}
时间复杂度:$O(n \log n)$.
\lstinputlisting{习题整理/错排后字典序最小(ZOJ4102).cpp}
\subsection{若干个区间选数字使相与之和最小(ZOJ4135)}
\lstinputlisting{习题整理/若干个区间选数字使相与之和最小(ZOJ4135).cpp}
\subsection{2019徐州L}
给一颗字符串树,1为根,求从哪个结点向上L长度的字符串共有多少种本质不同的字符串
\lstinputlisting{习题整理/2019徐州L.cpp}

%==============================习题整理==============================%
\section{Java \& Python}
eclipse下ALT+/,自动补全代码。
\subsection{Java}
\lstinputlisting[language=java]{xaveir/javaBasic.txt}
\lstinputlisting[language=java]{xaveir/javaExcrt.txt}
\subsection{Python}
\lstinputlisting[language=python]{xaveir/python.txt}

\end{document}