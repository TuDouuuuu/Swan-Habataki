\documentclass{article}
\usepackage{fontspec, xunicode, xltxtra} 
\setmainfont{Microsoft YaHei} 
\usepackage{setspace}
\usepackage{ctex} 
\usepackage{geometry}
\usepackage[Glenn]{fncychap}
\usepackage{listings}
\usepackage{color}
\usepackage{verbatim}
\usepackage{fancyhdr}


\newfontfamily\monaco{Monaco}

\definecolor{dkgreen}{rgb}{0,0.6,0}
\definecolor{gray}{rgb}{0.5,0.5,0.5}
\definecolor{mauve}{rgb}{0.58,0,0.82}

\geometry{left=2.5cm,right=1.7cm,top=1.0cm,bottom=1.5cm}

\lstset{
    frame=simple,
    language=c++,
    aboveskip=3mm,
    belowskip=3mm,
    showstringspaces=false,
    basicstyle=\monaco,
    numbers=none,
    numberstyle=\tiny\color{gray},
    keywordstyle=\bfseries\monaco,%\fontspec{monaco Bold}\bfseries,
    commentstyle=\color{dkgreen},
    stringstyle=\color{mauve},
    breaklines=true,
    breakatwhitespace=true,
    tabsize=4,
    numbers = left,
}

\begin{document}

\begin{titlepage}

\thispagestyle{empty}
\pagebreak
\pagestyle{plain}
\tableofcontents
\end{titlepage}

%----------------------STL--------------------
\section{STL}

\subsection{unordered\_map对pii进行哈希}
\lstinputlisting{STL/unordered_map对pii进行哈希.cpp}
\subsection{pb\_ds实现平衡树}
\lstinputlisting{STL/pb_ds实现平衡树.cpp}
\subsection{rope}
写一种数据结构,支持任意位置插入、删除和修改
\lstinputlisting{STL/rope.cpp}


%----------------------数据结构-------------------
\section{数据结构}
\subsection{单调栈}
\subsection{单调队列}
\subsection{笛卡尔树}
\subsection{树状数组}
\subsection{线段树}
\subsection{扫描线}
\subsection{可持久化线段树(主席树)}

\subsection{动态树LCT}
\subsubsection{}
\subsubsection{LCT+HJT主席树统计强连通分块个数}



\subsection{Splay}
\subsection{有旋Treap}
\subsubsection{普通平衡树(洛谷P3369)}
\lstinputlisting{数据结构/有旋Treap/普通平衡树Treap.cpp}
\subsection{无旋Treap(FHQ Treap)}


\subsection{可持久化FHQ}
第一行包含一个正整数$n$,表示操作的总数。\par
接下来$n$行,每行包含三个整数,第$i$行记为${v}_{i}$, ${opt}_i$, $x_i$。\par
$v_i$表示基于的过去版本号,${opt}_i$表示操作的序号,$x_i$表示参与操作的数值。
\lstinputlisting{数据结构/无旋Treap/可持久化FHQ.cpp}


\subsection{KD树}
\subsubsection{平面最近点对}
时间复杂度:单次查询最近点的时间复杂度$O(n).$
\lstinputlisting{数据结构/KD树/平面最近点对.cpp}
\subsubsection{K远点对([CQOI2016])}
已知平面内$N$个点的坐标,求欧氏距离下的第$K$远点对。\par
两个点的欧氏距离为$\sqrt{(x_1-x_2)^{2}+(y_1-y_2)^{2}}$\par
原题数据范围:$N\leq 1e5, 1\leq K\leq 100$\par
时间复杂度:$O(kn\log{n}).$
\lstinputlisting{数据结构/KD树/K远点对.cpp}
\subsubsection{高维空间上的操作}
在一个初始值全为$0$的$n\times n$的二维矩阵上,进行若干次操作,每次操作为以下两种之一:\par
\textbf{1 x y A} 将坐标$(x,y)$上的数加上$A$。\par
\textbf{2 x1 y1 x2 y2} 输出以$(x1, y1)$为左下角,$(x2, y2$为右上角的矩形内(包括矩形边界)的数字和。\par
原题数据范围:$1\leq n \leq 5e5, 1\leq q \leq 2e5$\par
时间复杂度:单次查询时间最优$O(\log{n})$, 最坏$O(\sqrt{n})$。将结论扩展至$k$维,最坏复杂度$O(n^{1-\frac{1}{k}})$
\lstinputlisting{数据结构/KD树/高维空间上的操作.cpp}


\subsection{珂朵莉树/老司机树/ODT}
\subsubsection{set实现珂朵莉树}
\textbf{1 l r x} 将$[l,r]$区间所有数加上$x$\par
\textbf{2 l r x} 将$[l,r]$区间所有数改成$x$\par
\textbf{3 l r x} 输出将$[l,r]$区间从小到大排序后的第$x$个数是的多少(即区间第$x$小,数字大小相同算多次,保证$1\leq x \leqn r-l+1$)\par
\textbf{4 l r x y} 输出$[l,r]$区间每个数字的$x$次方的和模$y$的值(即$\sum^r_{i=l}a_i^x \mod y $)\par
时间复杂度:用set实现$O(n\log \log {n})$\par
如果要保证复杂度正确,必须保证数据随机。\par
\lstinputlisting{数据结构/ODT/set实现珂朵莉树.cpp}
%----------------------字符串--------------------
\section{字符串}


\subsection{Next函数}
\subsubsection{求next函数}
\lstinputlisting{字符串/Next函数/求next函数.cpp}
\subsubsection{求出每个循环节的数量和终点位置(HDU1358)}
\lstinputlisting{字符串/Next函数/求出每个循环节的数量和终点位置(HDU1358).cpp}
\subsubsection{求同时是前缀和后缀的串长(POJ2752)}
\lstinputlisting{字符串/Next函数/求同时是前缀和后缀的串长(POJ2752).cpp}
\subsubsection{求字符串每个前缀和串匹配成功的次数和(HDU3336)}
\lstinputlisting{字符串/Next函数/求字符串每个前缀和串匹配成功的次数和(HDU3336).cpp}
\subsubsection{求循环节数量(POJ2406)}
\lstinputlisting{字符串/Next函数/求循环节数量(POJ2406).cpp}
\subsubsection{求第一个串的前缀和第二个串的后缀的最大匹配(HDU2594)}
\lstinputlisting{字符串/Next函数/求第一个串的前缀和第二个串的后缀的最大匹配(HDU2594).cpp}
\subsubsection{求补上最少字母数量使得这是个循环串(HDU3746)}
\lstinputlisting{字符串/Next函数/求补上最少字母数量使得这是个循环串(HDU3746).cpp}
\subsubsection{习题整理}
\textbf{[NOI2014]动物园}\par
对于字符串$S$的前i个字符构成的子串,既是它的后缀同时又是它的前缀,并且该后缀与该前缀不重叠,将这种字符串的数量记作$num[i]$.\par
$res$为$(num[i]+1)$的乘积.\par
时间复杂度:$O(n).$
\lstinputlisting{字符串/Next函数/习题整理.cpp}

\subsection{KMP}
\subsubsection{统计模式串出现次数,出现位置,前缀border长度}
\lstinputlisting{字符串/KMP/统计模式串出现次数,出现位置,前缀border长度.cpp}
\subsubsection{矩阵加速KMP,求长度为n的不包含长度为m的子串的串个数([HNOI2008]GT考试)}
$$\sum_{k=0}^{m-1}f[i-1][k]\ast g[k][j]$$\par
$f[i][j]$ 为长串匹配到第$i$位,短串最多可以匹配到第$j$位的方案数\par
$g[j][k]$ 为了计算长度为$j$的已经匹配好了的串可以用多少种数字变为$k$,枚举一个数字,看它在短串中最长可以匹配到最多多长的前缀\par
\lstinputlisting{字符串/KMP/矩阵加速KMP([HNOI2008]GT考试).cpp}

\subsection{EXKMP}
\subsubsection{求z函数和LCP}
$LCP$:最长公共前缀\par
$z$函数数组$z$:串$b$与$b$的每一个后缀的$LCP$长度。\par
$extend$数组:串$b$与串$a$的每一个后缀的$LCP$长度。
总时间复杂度:$O(|a|+|b|)$.
\lstinputlisting{字符串/EXKMP/求z函数和LCP.cpp}
\subsubsection{循环位移有多少数比原数大小相等,去重(HDU4333)}
包含对获得的串进行去重。\par
总时间复杂度:$O(n)$
\lstinputlisting{字符串/EXKMP/循环位移有多少数比原数大小相等,去重(HDU4333).cpp}


\subsection{AC自动机}
\subsubsection{标准的AC自动机}
\lstinputlisting{字符串/AC自动机.cpp}

\subsection{后缀自动机SAM}
\subsubsection{后缀自动机板子}
\textbf{应用1:不同子串个数}\par
给一个字符串$S$,计算不同子串的个数。\par
解法:利用后缀自动机的树形结构。每个节点对应的不同子串数量(不同位置算作同一个)是$maxlen[i]-maxlen[link[i]]$。\par
总时间复杂度:$O(|S|)$.\par
\textbf{应用2:所有不同子串的总长度}\par
给定一个字符串$S$,计算所有不同子串的总长度。
解法:利用上述后缀自动机的树形结构。每个节点对应的所有后缀长度是$\frac{maxlen[i]\ast (maxlen[i]+1)}{2}$,减去其$linke$节点的对应值$\frac{maxlen[link[i]]\ast (maxlen[link[i]]+1)}{2}$就是该节点的净贡献
\lstinputlisting{字符串/后缀自动机SAM/后缀自动机SAM.cpp}
\subsubsection{每个子串在多少个主串中出现过(SPOJ8093)}
暴力跳Link链.\par
时间复杂度:$均摊O(\sum |S|\sqrt{\sum |S|})$
\lstinputlisting{字符串/后缀自动机SAM/每个子串在多少个主串中出现过(SPOJ8093).cpp}


\subsection{回文自动机PAM}

\subsection{序列自动机([HEOI2015]最短不公共子串)}
时间复杂度:$O(n|\sum|)$,其中$|\sum|$为字符集大小
\lstinputlisting{字符串/序列自动机.cpp}

\subsection{最小表示法}
时间复杂度:$O(n)$
\lstinputlisting{字符串/最小表示法.cpp}


\end{document}