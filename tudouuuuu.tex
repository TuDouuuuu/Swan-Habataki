\documentclass{article}
\usepackage{fontspec, xunicode, xltxtra} 
\setmainfont{Microsoft YaHei} 
\usepackage{setspace}
\usepackage{ctex} 
\usepackage{geometry}
\usepackage[Glenn]{fncychap}
\usepackage{listings}
\usepackage{color}
\usepackage{verbatim}
\usepackage{fancyhdr}


\newfontfamily\monaco{Monaco}

\definecolor{dkgreen}{rgb}{0,0.6,0}
\definecolor{gray}{rgb}{0.5,0.5,0.5}
\definecolor{mauve}{rgb}{0.58,0,0.82}

\geometry{left=2.5cm,right=1.7cm,top=1.0cm,bottom=1.5cm}

\lstset{
    frame=simple,
    language=c++,
    aboveskip=3mm,
    belowskip=3mm,
    showstringspaces=false,
    basicstyle=\monaco,
    numbers=none,
    numberstyle=\tiny\color{gray},
    keywordstyle=\bfseries\monaco,%\fontspec{monaco Bold}\bfseries,
    commentstyle=\color{dkgreen},
    stringstyle=\color{mauve},
    breaklines=true,
    breakatwhitespace=true,
    tabsize=4,
    numbers = left,
}

\begin{document}

\begin{titlepage}

\thispagestyle{empty}
\pagebreak
\pagestyle{plain}
\tableofcontents
\end{titlepage}

%----------------------STL--------------------
\section{STL}

\subsection{unordered\_map对pii进行哈希}
\lstinputlisting{STL/unordered_map对pii进行哈希.cpp}
\subsection{pb\_ds实现平衡树}
\lstinputlisting{STL/pb_ds实现平衡树.cpp}
\subsection{rope}
写一种数据结构,支持任意位置插入、删除和修改
\lstinputlisting{STL/rope.cpp}


%----------------------数据结构-------------------
% \section{数据结构}
% \subsection{单调栈}
% \subsection{单调队列}

\subsection{ST表}
\lstinputlisting{数据结构/ST表.cpp}

\subsection{笛卡尔树}
\subsubsection{建树}
\lstinputlisting{数据结构/笛卡尔树/建树.cpp}


% \subsection{树状数组}
\subsection{线段树}
\subsubsection{区间中所有元素都严格出现三次的区间个数(CF1418G)}
\lstinputlisting{数据结构/线段树/普通线段树/区间中所有元素都严格出现三次的区间个数(CF1418G).cpp}
\subsubsection{线段树分裂合并}
\lstinputlisting{数据结构/线段树/普通线段树/线段树分裂合并.cpp}
\subsubsection{单点修改+单点最大连通数(HDU1540)}
\lstinputlisting{数据结构/线段树/普通线段树/单点修改单点最大连通个数.cpp}

% \subsection{扫描线}
\subsection{可持久化线段树(主席树)}
\subsubsection{静态区间第K小}
\lstinputlisting{数据结构/主席树/静态区间第K小.cpp}
\subsubsection{区间内不同数个数}
\lstinputlisting{数据结构/主席树/区间内不同数个数.cpp}
\subsubsection{静态区间第K小}
\lstinputlisting{数据结构/主席树/树上路径点权第K大.cpp}
\subsubsection{区间MEX}
\lstinputlisting{数据结构/主席树/区间MEX.cpp}

% \subsection{动态树LCT}
% \subsubsection{}
% \subsubsection{LCT+HJT主席树统计强连通分块个数}



% \subsection{Splay}
% \subsection{有旋Treap}
% \subsubsection{普通平衡树(洛谷P3369)}
% \lstinputlisting{数据结构/有旋Treap/普通平衡树Treap.cpp}
\subsection{无旋Treap(FHQ Treap)}
\subsubsection{区间翻转}
\lstinputlisting{数据结构/无旋Treap/fhq_treap.cpp}

\subsubsection{可持久化FHQ}
第一行包含一个正整数$n$,表示操作的总数。\par
接下来$n$行,每行包含三个整数,第$i$行记为${v}_{i}$, ${opt}_i$, $x_i$。\par
$v_i$表示基于的过去版本号,${opt}_i$表示操作的序号,$x_i$表示参与操作的数值。
\lstinputlisting{数据结构/无旋Treap/可持久化FHQ.cpp}


\subsection{树套树}
\subsubsection{带修主席树}
\lstinputlisting{数据结构/树套树/带修主席树.cpp}
\subsubsection{区间修改区间查询第K大([ZJOI2013]K大数查询)}
\lstinputlisting{数据结构/树套树/区间修改区间查询第K大([ZJOI2013]K大数查询).cpp}

\subsection{动态树}
\subsubsection{lct连链、断链、更改点权、查询链上点权异或和}
\lstinputlisting{数据结构/LCT/lct.cpp}
\subsubsection{树上路径染色}
\lstinputlisting{数据结构/LCT/树上路径染色.cpp}
\subsubsection{SAM+线段树+LCT离线统计区间本质不同字串个数}
时间复杂度:$O(n\log^{2}n+q\log n)$.
\lstinputlisting{数据结构/LCT/SAM+线段树+LCT离线统计区间本质不同字串个数.cpp}

\subsection{KD树}
\subsubsection{平面最近点对}
时间复杂度:单次查询最近点的时间复杂度$O(n).$
\lstinputlisting{数据结构/KD树/平面最近点对.cpp}
\subsubsection{K远点对([CQOI2016])}
已知平面内$N$个点的坐标,求欧氏距离下的第$K$远点对。\par
两个点的欧氏距离为$\sqrt{(x_1-x_2)^{2}+(y_1-y_2)^{2}}$\par
原题数据范围:$N\leq 1e5, 1\leq K\leq 100$\par
时间复杂度:$O(kn\log{n}).$
\lstinputlisting{数据结构/KD树/K远点对.cpp}
\subsubsection{高维空间上的操作}
在一个初始值全为$0$的$n\times n$的二维矩阵上,进行若干次操作,每次操作为以下两种之一:\par
\textbf{1 x y A} 将坐标$(x,y)$上的数加上$A$。\par
\textbf{2 x1 y1 x2 y2} 输出以$(x1, y1)$为左下角,$(x2, y2$为右上角的矩形内(包括矩形边界)的数字和。\par
原题数据范围:$1\leq n \leq 5e5, 1\leq q \leq 2e5$\par
时间复杂度:单次查询时间最优$O(\log{n})$, 最坏$O(\sqrt{n})$。将结论扩展至$k$维,最坏复杂度$O(n^{1-\frac{1}{k}})$
\lstinputlisting{数据结构/KD树/高维空间上的操作.cpp}


\subsection{珂朵莉树/老司机树/ODT}
\subsubsection{set实现珂朵莉树}
\textbf{1 l r x} 将$[l,r]$区间所有数加上$x$\par
\textbf{2 l r x} 将$[l,r]$区间所有数改成$x$\par
\textbf{3 l r x} 输出将$[l,r]$区间从小到大排序后的第$x$个数是的多少(即区间第$x$小,数字大小相同算多次,保证$1\leq x \leq r-l+1$)\par
\textbf{4 l r x y} 输出$[l,r]$区间每个数字的$x$次方的和模$y$的值(即$\sum ^ {r}_{i=l} a_i^x \% y$)\par
时间复杂度:用set实现$O(n\log \log {n})$\par
如果要保证复杂度正确,必须保证数据随机。\par
\lstinputlisting{数据结构/ODT/set实现珂朵莉树.cpp}
%----------------------字符串--------------------
\section{字符串}

\subsection{字符串哈希}
\subsubsection{区间一维哈希}
\lstinputlisting{字符串/Hash/Hash.cpp}
% \subsubsection{二维哈希}
% \lstinputlisting{}



\subsection{Next函数}
\subsubsection{求next函数}
\lstinputlisting{字符串/Next函数/求next函数.cpp}
\subsubsection{求出每个循环节的数量和终点位置(HDU1358)}
\lstinputlisting{字符串/Next函数/求出每个循环节的数量和终点位置(HDU1358).cpp}
\subsubsection{求同时是前缀和后缀的串长(POJ2752)}
\lstinputlisting{字符串/Next函数/求同时是前缀和后缀的串长(POJ2752).cpp}
\subsubsection{求字符串每个前缀和串匹配成功的次数和(HDU3336)}
\lstinputlisting{字符串/Next函数/求字符串每个前缀和串匹配成功的次数和(HDU3336).cpp}
\subsubsection{求循环节数量(POJ2406)}
\lstinputlisting{字符串/Next函数/求循环节数量(POJ2406).cpp}
\subsubsection{求第一个串的前缀和第二个串的后缀的最大匹配(HDU2594)}
\lstinputlisting{字符串/Next函数/求第一个串的前缀和第二个串的后缀的最大匹配(HDU2594).cpp}
\subsubsection{求补上最少字母数量使得这是个循环串(HDU3746)}
\lstinputlisting{字符串/Next函数/求补上最少字母数量使得这是个循环串(HDU3746).cpp}
\subsubsection{习题整理}
\textbf{[NOI2014]动物园}\par
对于字符串$S$的前i个字符构成的子串,既是它的后缀同时又是它的前缀,并且该后缀与该前缀不重叠,将这种字符串的数量记作$num[i]$.\par
$res$为$(num[i]+1)$的乘积.\par
时间复杂度:$O(n).$
\lstinputlisting{字符串/Next函数/习题整理.cpp}

\subsection{KMP}
\subsubsection{统计模式串出现次数,出现位置,前缀border长度}
\lstinputlisting{字符串/KMP/统计模式串出现次数,出现位置,前缀border长度.cpp}
\subsubsection{矩阵加速KMP,求长度为n的不包含长度为m的子串的串个数([HNOI2008]GT考试)}
$$\sum_{k=0}^{m-1}f[i-1][k]\ast g[k][j]$$\par
$f[i][j]$ 为长串匹配到第$i$位,短串最多可以匹配到第$j$位的方案数\par
$g[j][k]$ 为了计算长度为$j$的已经匹配好了的串可以用多少种数字变为$k$,枚举一个数字,看它在短串中最长可以匹配到最多多长的前缀\par
\lstinputlisting{字符串/KMP/矩阵加速KMP([HNOI2008]GT考试).cpp}

\subsection{EXKMP}
\subsubsection{求z函数和LCP}
$LCP$:最长公共前缀\par
$z$函数数组$z$:串$b$与$b$的每一个后缀的$LCP$长度。\par
$extend$数组:串$b$与串$a$的每一个后缀的$LCP$长度。
总时间复杂度:$O(|a|+|b|)$.
\lstinputlisting{字符串/EXKMP/求z函数和LCP.cpp}
\subsubsection{循环位移有多少数比原数大小相等,去重(HDU4333)}
包含对获得的串进行去重。\par
总时间复杂度:$O(n)$
\lstinputlisting{字符串/EXKMP/循环位移有多少数比原数大小相等,去重(HDU4333).cpp}


\subsection{AC自动机}
\subsubsection{标准的AC自动机}
\lstinputlisting{字符串/AC自动机.cpp}

\subsection{字典树/Trie树}
\lstinputlisting{字符串/Trie字典树.cpp}

% \subsection{序列自动机}
% \lstinputlisting{字符串/序列自动机.cpp}

% \subsection{最小表示法}
% \lstinputlisting{字符串/最小表示法.cpp}


\subsection{后缀数组SA}
\subsubsection{获取SA和rank数组}
\lstinputlisting{字符串/后缀数组/get_SA.cpp}
\subsubsection{后缀数组+ST表求lcp}
\lstinputlisting{字符串/后缀数组/后缀数组+ST表求lcp.cpp}


\subsection{后缀自动机SAM}
\subsubsection{后缀自动机板子}
\textbf{应用1:不同子串个数}\par
给一个字符串$S$,计算不同子串的个数。\par
解法:利用后缀自动机的树形结构。每个节点对应的不同子串数量(不同位置算作同一个)是$maxlen[i]-maxlen[link[i]]$。\par
总时间复杂度:$O(|S|)$.\par
\textbf{应用2:所有不同子串的总长度}\par
给定一个字符串$S$,计算所有不同子串的总长度。
解法:利用上述后缀自动机的树形结构。每个节点对应的所有后缀长度是$\frac{maxlen[i]\ast (maxlen[i]+1)}{2}$,减去其$linke$节点的对应值$\frac{maxlen[link[i]]\ast (maxlen[link[i]]+1)}{2}$就是该节点的净贡献
\lstinputlisting{字符串/后缀自动机SAM/后缀自动机SAM.cpp}
\subsubsection{每个子串在多少个主串中出现过(SPOJ8093)}
暴力跳Link链.\par
时间复杂度:$均摊O(\sum |S|\sqrt{\sum |S|})$
\lstinputlisting{字符串/后缀自动机SAM/每个子串在多少个主串中出现过(SPOJ8093).cpp}
\subsubsection{第k小字串}
\lstinputlisting{字符串/后缀自动机SAM/第k小字串(不同位置的相同子串算作一个&多个)([TJOI2015]弦论).cpp}
\subsubsection{字典树建后缀自动机}
\lstinputlisting{字符串/后缀自动机SAM/字典树建后缀自动机.cpp}
\subsubsection{暴力在线统计出现次数为k次的字符串个数(HDU4641)}
\lstinputlisting{字符串/后缀自动机SAM/暴力在线统计出现次数为k次的字符串个数(HDU4641).cpp}


% \subsection{Manacher}


\subsection{回文自动机PAM}
\lstinputlisting{字符串/回文自动机/回文自动机PAM.cpp}


\subsection{序列自动机([HEOI2015]最短不公共子串)}
时间复杂度:$O(n|\sum|)$,其中$|\sum|$为字符集大小
\lstinputlisting{字符串/序列自动机.cpp}

\subsection{最小表示法}
时间复杂度:$O(n)$
\lstinputlisting{字符串/最小表示法.cpp}

\subsection{Lyndon分解}
将字符串分成若干部分$s = s_{1}s_{2}s_{3}...s_{m}$,使得每个$s_{i}$都是$Lyndon Word$。\par
$Lyndon Word$:当且仅当$s$是其所有后缀中最小字符串。
\lstinputlisting{字符串/Lyndon分解.cpp}

%----------------------动态规划--------------------
% \section{动态规划}
% \subsection{背包DP}

% \subsection{区间DP}

% \subsection{DAG上DP}

% \subsection{数型DP}


%----------------------杂项-------------------
\section{杂项}
\subsection{LCA}
\lstinputlisting{杂项/LCA.cpp}
% \subsection{三分}

\subsection{CDQ}
\subsubsection{三维偏序}
有$n$个元素,第$i$个元素有$a_i$,$b_i$,$c_i$三个属性,设$f(i)$表示满足$a_j \leq a_i$且$b_j \leq b_i$且$c_j \leq c_i$且$j \ne i$的$j$的数量。\par
对于$d\in [0, n)$,求$f(i)=d$的数量。
\lstinputlisting{杂项/CDQ/三位偏序.cpp}

% \subsection{莫队}

% \subsection{格雷码}

% \subsection{可撤销并查集}


\end{document}