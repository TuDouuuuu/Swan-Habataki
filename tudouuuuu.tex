\documentclass{article}
\usepackage{fontspec, xunicode, xltxtra} 
\setmainfont{Microsoft YaHei} 
\usepackage{setspace}
\usepackage{ctex} 
\usepackage{geometry}
\usepackage[Glenn]{fncychap}
\usepackage{listings}
\usepackage{color}
\usepackage{verbatim}
\usepackage{fancyhdr}


\newfontfamily\monaco{Monaco}

\definecolor{dkgreen}{rgb}{0,0.6,0}
\definecolor{gray}{rgb}{0.5,0.5,0.5}
\definecolor{mauve}{rgb}{0.58,0,0.82}

\geometry{left=2.5cm,right=1.7cm,top=1.0cm,bottom=1.5cm}

\lstset{
    frame=simple,
    language=c++,
    aboveskip=3mm,
    belowskip=3mm,
    showstringspaces=false,
    basicstyle=\monaco,
    numbers=none,
    numberstyle=\tiny\color{gray},
    keywordstyle=\bfseries\monaco,%\fontspec{monaco Bold}\bfseries,
    commentstyle=\color{dkgreen},
    stringstyle=\color{mauve},
    breaklines=true,
    breakatwhitespace=true,
    tabsize=4,
    numbers = left,
}

\begin{document}

\begin{titlepage}

\thispagestyle{empty}
\pagebreak
\pagestyle{plain}
\tableofcontents
\end{titlepage}

%----------------------字符串--------------------
\section{字符串}

\subsection{Next函数}
\subsubsection{求next函数}
\lstinputlisting{字符串/Next函数/求next函数.cpp}
\subsubsection{求出每个循环节的数量和终点位置(HDU1358)}
\lstinputlisting{字符串/Next函数/求出每个循环节的数量和终点位置(HDU1358).cpp}
\subsubsection{求同时是前缀和后缀的串长(POJ2752)}
\lstinputlisting{字符串/Next函数/求同时是前缀和后缀的串长(POJ2752).cpp}
\subsubsection{求字符串每个前缀和串匹配成功的次数和(HDU3336)}
\lstinputlisting{字符串/Next函数/求字符串每个前缀和串匹配成功的次数和(HDU3336).cpp}
\subsubsection{求循环节数量(POJ2406)}
\lstinputlisting{字符串/Next函数/求循环节数量(POJ2406).cpp}
\subsubsection{求第一个串的前缀和第二个串的后缀的最大匹配(HDU2594)}
\lstinputlisting{字符串/Next函数/求第一个串的前缀和第二个串的后缀的最大匹配(HDU2594).cpp}
\subsubsection{求补上最少字母数量使得这是个循环串(HDU3746)}
\lstinputlisting{字符串/Next函数/求补上最少字母数量使得这是个循环串(HDU3746).cpp}
\subsubsection{习题整理}
\textbf{[NOI2014]动物园}\par
对于字符串$S$的前i个字符构成的子串,既是它的后缀同时又是它的前缀,并且该后缀与该前缀不重叠,将这种字符串的数量记作$num[i]$.\par
$res$为$(num[i]+1)$的乘积.\par
时间复杂度:$O(n).$
\lstinputlisting{字符串/Next函数/习题整理.cpp}

\subsection{KMP}
\subsubsection{统计模式串出现次数,出现位置,前缀border长度}
\lstinputlisting{字符串/KMP/统计模式串出现次数,出现位置,前缀border长度.cpp}
\subsubsection{矩阵加速KMP,求长度为n的不包含长度为m的子串的串个数([HNOI2008]GT考试)}
$$\sum_{k=0}^{m-1}f[i-1][k]\ast g[k][j]$$\par
$f[i][j]$ 为长串匹配到第$i$位,短串最多可以匹配到第$j$位的方案数\par
$g[j][k]$ 为了计算长度为$j$的已经匹配好了的串可以用多少种数字变为$k$,枚举一个数字,看它在短串中最长可以匹配到最多多长的前缀\par
\lstinputlisting{字符串/KMP/矩阵加速KMP([HNOI2008]GT考试).cpp}

\subsection{EXKMP}

\subsection{AC自动机}
\subsubsection{标准的AC自动机}
\lstinputlisting{字符串/AC自动机.cpp}

\subsection{后缀自动机SAM}
\subsubsection{后缀自动机板子}
\textbf{应用1:不同子串个数}\par
给一个字符串$S$,计算不同子串的个数。\par
解法:利用后缀自动机的树形结构。每个节点对应的不同子串数量(不同位置算作同一个)是$maxlen[i]-maxlen[link[i]]$。\par
总时间复杂度:O(|S|).
\textbf{应用2:所有不同子串的总长度}\par
给定一个字符串$S$,计算所有不同子串的总长度。
解法:利用上述后缀自动机的树形结构。每个节点对应的所有后缀长度是$\frac{maxlen[i]\ast (maxlen[i]+1)}{2}$,减去其$linke$节点的对应值$\frac{maxlen[link[i]]\ast (maxlen[link[i]]+1)}{2}$就是该节点的净贡献

\lstinputlisting{字符串/后缀自动机SAM/后缀自动机SAM.cpp}

\end{document}