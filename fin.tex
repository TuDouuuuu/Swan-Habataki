%==============================常用宏包、环境==============================%
\documentclass[twoside,a4paper]{article}
\usepackage{xeCJK} % For Chinese characters
\usepackage{amsmath, amsthm}
\usepackage{listings,xcolor}
\usepackage{geometry} % 设置页边距
\usepackage{fontspec}
\usepackage{graphicx}
\usepackage{fancyhdr} % 自定义页眉页脚
\usepackage{makecell}
\usepackage[breaklinks,colorlinks,linkcolor=black,citecolor=black,urlcolor=black]{hyperref}
\setsansfont{Consolas} % 设置英文字体
\setmonofont[Mapping={}]{Consolas} % 英文引号之类的正常显示,相当于设置英文字体
\geometry{left=1.8cm,right=1cm,top=1.5cm,bottom=0.5cm} % 页边距
% \setlength{\columnsep}{15pt}
% \setlength\columnseprule{0.4pt} % 分割线

\usepackage{xunicode, xltxtra} 
\setmainfont{Microsoft YaHei} 
\usepackage{setspace}
\usepackage{ctex} 
\usepackage[Glenn]{fncychap}
\usepackage{color}
\usepackage{verbatim}
\usepackage{titlesec}
\usepackage{markdown}
 

%==============================常用宏包、环境==============================%
\newfontfamily\monaco{Monaco}
\definecolor{dkgreen}{rgb}{0,0.6,0}
\definecolor{gray}{rgb}{0.5,0.5,0.5}
\definecolor{mauve}{rgb}{0.58,0,0.82}

%==============================页眉、页脚、代码格式设置==============================%
% 页眉、页脚设置

\pagestyle{fancy}
\fancyhead{} %clear all fields
\fancyhead[RO]{\CJKfamily{hei} 第 \thepage 页} %奇数页眉的右边
\fancyhead[LE]{\CJKfamily{hei} 第 \thepage 页} %奇数页眉的右边
\renewcommand{\headrulewidth}{0.4pt} 
\renewcommand{\footrulewidth}{0.4pt}

% \lstset{
%     language    = c++,
%     numbers     = left,
%     numberstyle = \tiny,
%     breaklines  = true,
%     captionpos  = b,
%     tabsize     = 4,
%     frame       = simple,
%     columns     = fullflexible,
%     commentstyle = \color{gray},
%     keywordstyle = \bfseries\monaco,
%     basicstyle   = \monaco,
%     stringstyle  = \color[RGB]{148,0,209}\ttfamily,
%     rulesepcolor = \color{red!20!green!20!blue!20},
%     showstringspaces = false,
% }

\lstset{
    frame = simple,
    language = c++,
    aboveskip = 3mm,
    belowskip = 3mm,
    showstringspaces = false,
    basicstyle = \monaco,
    numbers = left,
    numberstyle = \tiny\color{gray},
    keywordstyle = \bfseries\monaco,%\fontspec{monaco Bold}\bfseries,
    commentstyle = \color{gray},
    stringstyle = \color{mauve},
    breaklines = true,
    breakatwhitespace = true,
    tabsize = 4
}

%==============================页眉、页脚、代码格式设置==============================%

%==============================标题和目录==============================%
% \title{\CJKfamily{hei} \bfseries Standard Code Library}
% \author{Xavier\_Cai}
% \renewcommand{\today}{\number\year 年 \number\month 月 \number\day 日}

\begin{document}\small
% \begin{titlepage}
% \maketitle
% \end{titlepage}

\newpage
\pagestyle{empty}
\renewcommand{\contentsname}{目录}
\tableofcontents
\newpage\clearpage
\newpage

\newpage
\mbox{}
\newpage


\pagestyle{fancy}
\setcounter{page}{1}   %new page


%==============================数据结构=============================%
\section{数据结构}

\subsection{单调栈}
\lstinputlisting{数据结构/单调栈_单调队列/单调栈.cpp}

\subsection{单调队列}
\lstinputlisting{数据结构/单调栈_单调队列/单调队列.cpp}

\subsection{线段树套平衡树}
\textbf{1 l r k} 查询k在区间内的排名 $O(\log ^{2} N)$ \par
\textbf{2 l r k} 查询区间内排名为k的值 $O(\log ^{3} N)$ \par
\textbf{3 pos k} 修改某一位值上的数值 $O(\log ^{2} N)$\par
\textbf{4 l r k} 查询k在区间内的前驱(前驱定义为严格小于x,且最大的数,若不存在输出-2147483647) $O(\log ^{3} N)$ \par
\textbf{5 l r k} 查询k在区间内的后继(后继定义为严格大于x,且最小的数,若不存在输出2147483647) $O(\log ^{3} N)$ \par
后面两个也有 $O(\log ^{2} N)$ 的做法,求前驱/后继后取 max, min。 \par
\lstinputlisting{数据结构/树套树/线段树套平衡树.cpp}

%==============================图论=============================%
\section{图论}

\subsection{点分治}
\subsubsection{两点间的距离是否为3的倍数([国家集训队]聪聪可可)}
\lstinputlisting{图论/点分治/两点间距离是否为3的倍数.cpp}

\subsubsection{询问树上距离为k的点对是否存在}
时间复杂度:$O(n \log^{2} n + nm \log n)$
\lstinputlisting{图论/点分治/询问树上距离为k的点对是否存在.cpp}

\subsubsection{两点间距离不超过l距离权重和不超过w}
\lstinputlisting{图论/点分治/两点间距离不超过l距离权重和不超过w.cpp}

\subsection{DSU}
\subsubsection{询问子树颜色种类数}
\lstinputlisting{图论/DSU/询问子树颜色种类数.cpp}

\subsection{无向图相邻边成对}
\lstinputlisting{图论/无向图相邻边成对.cpp}

\subsection{BFS TREE(CF1496F)}
给定一张无向图,取任意两点进行如下操作: \par
以这两点 $x,y$ 为源构造生成树,使得对于任意点k,有$dis[x][k]=min(dis[x][k])\&\&dis[y][k]=min(dis[y][k])$ \par
即 $x,y$ 点与 $k$ 点的距离即为 $x,y$ 与 $k$ 的最短路径(之一),对于每一对 $x,y$ 求能构造的生成树有多少 \par

\par
记 $dist(x,y)$ 为从点 $x$ 到点 $y$ 所经过的点的个数。 \par
有两点性质: \par
1. 对于点 $z$,当 $dist(x,z)+dist(y,z)-1=dist(x,y)$,那么点 $z$ 应当是在从 $x$ 到 $y$ 的最短路上。
  特别的,当这样的点的个树超过 $dist(x,y)$ 个时,那么 $x$ 和 $y$ 作为根节点的BFS树同构必定不存在。 \par
2. 对于其他不在从 $x$ 到 $y$ 的最短路的点 $u$,要存在相邻的点 $v$,使得 $dist(x,v)=dist(x,u)-1$ 并且 $dist(y,v)=dist(y,u)-1$。 \par

\lstinputlisting{习题整理2/BFS.cpp}


%==============================整体二分=============================%
\section{整体二分}

\subsection{每次询问一个子矩阵的第 $k$ 小数}
\lstinputlisting{整体二分/子矩阵第k大.cpp}

\subsection{带修主席树}
\lstinputlisting{整体二分/带修主席树.cpp}

\subsection{在右半边的整体二分}
\lstinputlisting{整体二分/在右半边的整体二分.cpp}


%==============================几何=============================%
\section{计算几何}

\subsection{凸包}
\lstinputlisting{yx计算几何/凸包.cpp}

\subsection{四边形}
\subsubsection{四边形计数}
\lstinputlisting{yx计算几何/四边形计数.cpp}

\subsubsection{四边形最小面积计数}
\lstinputlisting{yx计算几何/四边形最小面积计数.cpp}


%==============================习题整理=============================%
\section{习题整理}
\subsection{dfs+2019银川A}
\lstinputlisting{2019银川/A.cpp}

\subsection{链哈LCA倍增}
\lstinputlisting{习题整理2/链哈LCA倍增.cpp}

\subsection{后缀自动机+set}
给定一个长度为 $n$ 的字符串,对于前缀 $1..i$,找到最短的字符串,使其在整个长度为 $n$ 的字符串中只出现一次,输出长度。 \par
对于后缀自动机,其实是有三种情况进行分类讨论的:\par
1. 直接连到起始节点,这种情况不会出现重复。 \par
2. 直连,类似 aa,这种情况会出现重复。 \par
3. 之前出现过的状态,需要复制节点信息,这种情况会出现重复。 \par
\lstinputlisting{习题整理2/sam_set.cpp}

\subsection{CF840D Destiny}
给定 $n$ 个元素,$m$ 次询问。 \par
每次给出三个参数 $l,r,k$,询问区间 $[l,r]$ 内是否存在出现次数严格大于 $\frac{r-l+1}{k}$ 的数。如果存在就输出最小的那个 $ans$ ,否则输出 $-1$. \par
时间复杂度: $O(nk \log n)$
\lstinputlisting{习题整理2/cf840.cpp}

\end{document}